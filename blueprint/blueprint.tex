\documentclass{report}

\usepackage{amsmath, amsthm}
\usepackage[showmore, dep_graph, coverage, project=../../]{blueprint}

\theoremstyle{definition}
\newtheorem{definition}{Definition}
\newtheorem{theorem}{Theorem}
\newtheorem{proposition}{Proposition}
\newtheorem{lemma}{Lemma}
\newtheorem{corollary}{Corollary}

\dochome{https://ianjauslin-rutgers.github.io/lean-complex-has_primitives/docs}

\title{Complex Analysis: Has Primitives}
\author{Rutgers Lean Seminar}

%
% This is a blueprint for the HasPrimitives project, written by the Rutgers Lean Group
% 
% This blueprint can be edited like a standard LaTeX document.
%
% There are 3 useful commands that are specific to lean:
%   * \lean{leanref}: is used in a definition, lemma or theorem, and is the
%		      name of the lean object corresponding to the definition,
%		      lemma or theorem.
%   * \leanok: is used in a lemma or theorem: marks that the lemma or theorem
%              has been fully formalized
%   * \uses{ref}: is used in a proof environment: specifies which definitions,
%                 lemma, theorems are used in the proof. Note that 'ref' is a
%                 LaTeX ref, not a lean one.
%

\begin{document}
\maketitle

This project aims to formalize a proof that holomorphic functions on simply connected open sets have primitives.

\chapter{The project}

\section{Main result}


We define the notion of a primitive as follows.

\begin{definition}
  \label{hasPrimitives}
  \lean{hasPrimitives}\leanok
  Given a set $U\subset\mathbb C$, for any differentiable $f:U\to\mathbb C$, there exists a differentiable $g:U\to\mathbb C$ such that $g'=f$ on $U$.
\end{definition}

The main result is

\begin{theorem}
  \label{hasPrimitivesOfSimplyConnected}
  \lean{hasPrimitivesOfSimplyConnected}\leanok
  \uses{hasPrimitives}
  Given an open, simply connected set $U\subset\mathbb C$, $U$ has primitives, in the sense of Definition \ref{hasPrimitives}.
\end{theorem}

\begin{proof}
  \uses{hasPrimitivesOfConvex}
  sorry
\end{proof}

\section{Primitives on a convex set}
The idea of the proof of the main theorem is to first prove it on a convex set, and then to extend to all simply connected sets.

\begin{theorem}
  \label{hasPrimitivesOfConvex}
  \lean{hasPrimitivesOfConvex}\leanok
  \uses{hasPrimitives}
  An open convex set has primitives, in the sense of Definition \ref{hasPrimitives}.
\end{theorem}

\begin{proof}
  \uses{linint,diffOfIntegrals,derivOfLinint}
  Consider a convex set $U$ and a point $z_0\in U$.
  Given a function $f$ that is holomorphic on $U$, we construct its primitive by integrating $f$ over a line from $z_0$ to $z$: for $z\in U$,
  \begin{equation}
    g(z)=\int_{z_0\to z}\ f(z)\ dz
  \end{equation}
  where $\int_{z_0\to z}$ is defined in Definition \ref{linint}.
  We are then left with proving that $g$ is differentiable and that $g'=f$.

  To do so, we compute
  \begin{equation}
    \frac{g(z+h)-g(z)}h
    =
    \frac{\int_{z_0\to z+h} f(x)\ dx-\int_{z_0\to z} f(x)\ dx}h
  \end{equation}
  which by Lemma \ref{diffOfIntegrals}, is
  \begin{equation}
    \frac{g(z+h)-g(z)}h
    =
    \frac{\int_{z\to z+h} f(x)\ dx}h
    .
  \end{equation}
  We then conclude using Lemma \ref{derivOfLinint}.
\end{proof}

The integral over the line from $z_1$ to $z_2$ is defined as follows.

\begin{definition}
  \label{linint}
  \lean{linint}\leanok
  Given $z_1,z_2\in\mathbb C$ and a function $f:\mathbb C\to\mathbb C$, the path from $z_1$ to $z_2$ is defined as
  \begin{equation}
    \gamma(t):=z_1(1-t)+z_2 t
  \end{equation}
  and $\int_{z_1\to z_2}$ is defined as the complex integral along this path.
\end{definition}

In the proof of Theorem \ref{hasPrimitivesOfConvex}, we use the following lemmas.

\begin{lemma}
  \label{diffOfIntegrals}
  \lean{diffOfIntegrals}\leanok
  \uses{linint}
  Given a convex open set $U$, $z_0,z_1,z_2\in U$ and a function $f$ that is holomorphic on $U$
  \begin{equation}
    \int_{z_0\to z_1} f(x)\ dx
    -
    \int_{z_0\to z_2} f(x)\ dx
    =
    \int_{z_2\to z_1} f(x)\ dx
    .
  \end{equation}
\end{lemma}

\begin{proof}
  \uses{cauchyOnTriangles}
  Consider the path $z_0\to z_1\to z_2$.
  Since $U$ is convex, the area enclosed by this path is a subset of $U$.
  Therefore, by Theorem \ref{cauchyOnTriangles}, the integral of $f$ over this path is 0.
  We conclude the proof by splitting the path into its components.
\end{proof}

\begin{lemma}
  \label{derivOfLinint}
  \lean{derivOfLinint}\leanok
  \uses{linint}
  Given a continuous function $f:\mathbb C\to\mathbb C$ and $z_0\in\mathbb C$,
  \begin{equation}
    \lim_{h\to0}\frac1h\int_{z_0\to z_0+h}f(x)\ dx=f(z_0)
    .
  \end{equation}
\end{lemma}

\begin{proof}
  Since $f$ is continuous,
  \begin{equation}
    \int_{z_0\to z_0+h}f(x)\ dx
    =
    \int_{z_0\to z_0+h}f(z_0)+o(1)\ dx
    =
    hf(z_0)+o(h)
    .
  \end{equation}
\end{proof}

To prove Lemma \ref{diffOfIntegrals}, we use a version of Cauchy's theorem on triangles, which we prove here.

\begin{theorem}
  \label{cauchyOnTriangles}
  \uses{linint}
  Given $a,b,c\in\mathbb C$, and a function $f$ that is holomorphic on the triangle formed by $a,b,c$,
  \begin{equation}
    \int_{a\to b} f(x)\ dx
    +
    \int_{b\to c} f(x)\ dx
    +
    \int_{c\to a} f(x)\ dx
    =0
    .
  \end{equation}
\end{theorem}

\begin{proof}
  \uses{cauchyOnTriangle_base}
  We translate $a$ and $b$ to the real line, and use Lemma \ref{cauchyOnTriangle_base}.
\end{proof}

\begin{lemma}
  \label{cauchyOnTriangle_base}
  \uses{linint}
  Given $a,b\in\mathbb R$, $c\in\mathbb C$, and a function $f$ that is holomorphic on the triangle formed by $a,b,c$,
  \begin{equation}
    \int_{a\to b} f(x)\ dx
    +
    \int_{b\to c} f(x)\ dx
    +
    \int_{c\to a} f(x)\ dx
    =0
    .
  \end{equation}
\end{lemma}

\begin{proof}
  \uses{cauchyOnRightTriangle}
  Let $d:=\mathcal Re(c)$.
  Consider the triangles $(d,b,c)$ and $(d,a,c)$.
  We have
  \begin{equation}
    \int_{a\to b} f(x)\ dx
    +
    \int_{b\to c} f(x)\ dx
    +
    \int_{b\to a} f(x)\ dx
    =
    \int_{a\to d} f(x)\ dx
    +
    \int_{d\to c} f(x)\ dx
    +
    \int_{c\to a} f(x)\ dx
    +
    \int_{d\to b} f(x)\ dx
    +
    \int_{b\to c} f(x)\ dx
    +
    \int_{c\to d} f(x)\ dx
  \end{equation}
  and we compute each groups of three using Lemma \ref{cauchyOnRightTriangle}.
\end{proof}

\begin{lemma}
  \label{cauchyOnRightTriangle}
  \uses{linint}
  Given $a,b\in\mathbb R$ and $z\in\mathbb C$, and a function $f$ that is holomorphic on the triangle formed by $z,z+a,z+ib$,
  \begin{equation}
    \int_{z\to z+a} f(x)\ dx
    +
    \int_{z+a\to z+ib} f(x)\ dx
    +
    \int_{z+ib\to z} f(x)\ dx
    =0
    .
  \end{equation}
\end{lemma}

\begin{proof}
  \uses{cauchyOnRightTriangle_origin}
  We shift the points by $z$ and use Lemma \ref{cauchyOnRightTriangle_origin}.
\end{proof}

\begin{lemma}
  \label{cauchyOnRightTriangle_origin}
  \uses{linint}
  Given $a,b\in\mathbb R$, and a function $f$ that is holomorphic on the triangle formed by $0,a,ib$,
  \begin{equation}
    \int_{0\to a} f(x)\ dx
    +
    \int_{a\to ib} f(x)\ dx
    +
    \int_{ib\to 0} f(x)\ dx
    =0
    .
  \end{equation}
\end{lemma}

\begin{proof}
  \uses{cauchyRiemann}
  Let $u:=\mathcal Re(f)$ and $v:=\mathcal Im(f)$.
  We have
  \begin{equation}
    \int_{0\to a} f(x)\ dx
    +
    \int_{a\to ib} f(x)\ dx
    +
    \int_{ib\to 0} f(x)\ dx
    =
    \int_0^a f(x)\ dx
    -\int_0^b if(iy)\ dy
    +
    \int_0^b f(a+y(i-a/b))(i-a/b)\ dy
    .
  \end{equation}
  Furthermore,
  \begin{equation}
    \int_0^b f(a+y(i-a/b))\ dy
    -\int_0^b f(iy)\ dy
    =
    \int_0^b\int_0^{a-ya/b}\ \partial_x f(x+iy)dxdy
  \end{equation}
  so
  \begin{equation}
    \int_0^a f(x)\ dx
    -\int_0^b if(iy)\ dy
    +
    \int_0^b f(a+y(i-a/b))(i-a/b)\ dy
    =
    i\int_0^b\int_0^{a-ya/b}\ \partial_x f(x+iy)dxdx
    -
    \frac ab
    \int_0^b f(a+y(i-a/b))\ dy
    +
    \int_0^a f(x)\ dx
    .
  \end{equation}
  Now, changing variables to $x=a-ya/b$,
  \begin{equation}
    \frac ab
    \int_0^b f(a+y(i-a/b))\ dy
    =
    \int_0^a f(ib+x(1-ib/a))\ dx
  \end{equation}
  and
  \begin{equation}
    \int_0^a f(ib+x(1-ib/a))\ dx
    -
    \int_0^a f(x)\ dx
    =
    \int_0^a\int_0^{b-xb/a)}\partial_yf(x+iy)\ dydx
    .
  \end{equation}
  All in all,
  \begin{equation}
    \int_{0\to a} f(x)\ dx
    +
    \int_{a\to ib} f(x)\ dx
    +
    \int_{ib\to 0} f(x)\ dx
    =
    i\int_0^b\int_0^{a-ya/b}\ \partial_x f(x+iy)dxdy
    -
    \int_0^a\int_0^{b-xb/a)}\partial_yf(x+iy)\ dydx
    .
  \end{equation}
  That is,
  \begin{equation}
    \int_{0\to a} f(x)\ dx
    +
    \int_{a\to ib} f(x)\ dx
    +
    \int_{ib\to 0} f(x)\ dx
    =
    i\int_0^b\int_0^{a-ya/b}\ (\partial_x u-\partial_y v)dxdy
    -
    \int_0^a\int_0^{b-xb/a)}(\partial_yu+\partial_x v)\ dydx
    .
  \end{equation}

  Now by Theorem \ref{cauchyRiemann},
  \begin{equation}
    \partial_x u=\partial_y v
    ,\quad
    \partial_y u=-\partial_x v
  \end{equation}
  so
  \begin{equation}
    \int_{0\to a} f(x)\ dx
    +
    \int_{a\to ib} f(x)\ dx
    +
    \int_{ib\to 0} f(x)\ dx
    =
    0
    .
  \end{equation}
\end{proof}

\begin{theorem}
  \label{cauchyRiemann}
  If $f:\mathbb C\to\mathbb C$ is differentiable at $z\in\mathbb C$, then, denoting $z=x+iy$,
  \begin{equation}
    \partial_x\mathcal Re(f(z))=\partial_y\mathcal Im(f(z))
    ,\quad
    \partial_y\mathcal Re(f(z))=-\partial_x\mathcal Im(f(z))
  \end{equation}
\end{theorem}

\begin{proof}
  Since $f$ is differentiable, for $a\in\mathbb R$,
  \begin{equation}
    \lim_{a\to0}\frac{f(z+a)-f(z)}a=\frac{f(z+ia)-f(z)}{ia}
  \end{equation}
  that is
  \begin{equation}
    \partial_x f(z)=-i\partial_y f(z)
  \end{equation}
  from which we conclude.
\end{proof}


\end{document}
