\documentclass{report}

\usepackage{amsmath, amsthm}
\usepackage[showmore, dep_graph, coverage, project=../../]{blueprint}

\theoremstyle{definition}
\newtheorem{definition}{Definition}
\newtheorem{theorem}{Theorem}
\newtheorem{proposition}{Proposition}
\newtheorem{lemma}{Lemma}
\newtheorem{corollary}{Corollary}

\dochome{https://ianjauslin-rutgers.github.io/lean-complex-has_primitives/docs}

\title{Complex Analysis: Has Primitives}
\author{Rutgers Lean Seminar}

%
% This is a blueprint for the HasPrimitives project, written by the Rutgers Lean Group
% 
% This blueprint can be edited like a standard LaTeX document.
%
% There are 3 useful commands that are specific to lean:
%   * \lean{ref}: is used in a definition, lemma or theorem, and is the name of
%                 the lean object corresponding to the definition, lemma or theorem.
%   * \leanok: is used in a lemma or theorem: marks that the lemma or theorem
%              has been fully formalized
%   * \uses{ref}: is used in a proof environment: specifies which lean objects
%                 are used in the proof
%

\begin{document}
\maketitle

This project aims to formalize a proof that holomorphic functions on simply connected open sets have primitives.

\chapter{The project}

\section{Main result}

We define the notion of a primitive as follows.

\begin{definition}
  \label{def:hasPrimitives}
  \lean{hasPrimitives}\leanok
  Given a set $U\subset\mathbb C$, for any differentiable $f:U\to\mathbb C$, there exists a differentiable $g:U\to\mathbb C$ such that $f'=g$ on $U$.
\end{definition}

The main result is
\begin{theorem}
  \label{theo:hasPrimitivesOfSimplyConnected}
  \lean{hasPrimitivesOfSimplyConnected}\leanok
  Given an open, simply connected set $U\subset\mathbb C$, $U$ has primitives, in the sense of Definition \ref{def:hasPrimitives}.
\end{theorem}

\begin{proof}
  \uses{theo:hasPrimitivesOfDisc}
\end{prood}

\section{Primitives on a disc}
The idea of the proof of the main theorem is to first prove it on a disc, and then to extend to all simply connected sets.

The main result on the unit disc is
\begin{theorem}
  \label{theo:hasPrimitivesOfDisc}
  \lean{hasPrimitivesOfDisc}\leanok
  The unit complex disc has primitives, in the sense of Definition \ref{def:hasPrimitives}.
\end{theorem}

\begin{proof}
  \uses{linint,diffOfIntegrals,derivOfLinint}
  Given a function $f$ that is holomorphic on the unit disc, we construct its primitive by integrating $f$ over a line from the origin to $z$: for $z\in\mathbb C$,
  \begin{equation}
    g(z)=\int_{0\to z}\ f(z)\ dz
  \end{equation}
  where $\int_{0\to z}$ is defined in Definition \ref{def:linint}.
  We are then left with proving that $g$ is differentiable and that $g'=f$.

  To do so, we compute
  \begin{equation}
    \frac{g(z+h)-g(z)}h
    =
    \frac{\int_{0\to z+h} f(x)\ dx-\int_{0\to z} f(x)\ dx}h
  \end{equation}
  which by Lemma \ref{lemma:diffOfIntegrals}, is
  \begin{equation}
    \frac{g(z+h)-g(z)}h
    =
    \frac{\int_{z\to z+h} f(x)\ dx}h
    .
  \end{equation}
  We then conclude using Lemma \ref{lemma:derivOfLinint}.
\end{proof}

The integral over the line from $z_1$ to $z_2$ is defined as follows.

\begin{definition}
  \label{def:linint}
  \lean{linint}\leanok
  \uses{straightSeg,curvint}
  Given $z_1,z_2\in\mathbb C$ and a function $f:\mathbb C\to\mathbb C$, the path from $z_1$ to $z_2$ is defined as
  \begin{equation}
    \gamma(t):=z_1(1-\cos(t\pi))+z_2\cos(t\pi)
  \end{equation}
  and $\int_{z_1\to z_2}$ is defined as the complex integral along this path.
\end{definition}

In the proof of Theorem \ref{theo:hasPrimitivesOfDisc}, we use the following lemmas.

\begin{lemma}
  \label{lemma:diffOfIntegrals}
  \lean{diffOfIntegrals}\leanok
  Given $z_0,z_1,z_2$ in the unit complex disc and a function $f$ that is holomorphic on the unit complex disc,
  \begin{equation}
    \int_{z_0\to z_1} f(x)\ dx
    -
    \int_{z_0\to z_2} f(x)\ dx
    =
    \int_{z_2\to z_1} f(x)\ dx
    .
  \end{equation}
\end{lemma}

\begin{proof}
\end{proof}

\begin{lemma}
  \label{lemma:derivOfLinint}
  \lean{derivOfLinint}\leanok
  Given a continuous function $f:\mathbb C\to\mathbb C$ and $z_0\in\mathbb C$,
  \begin{equation}
    \lim_{h\to0}\frac{\int_{z_0\to z_0+h}f(x)\ dx}h=f(z_0)
    .
  \end{equation}
\end{lemma}

\begin{proof}
\end{proof}

\end{document}
